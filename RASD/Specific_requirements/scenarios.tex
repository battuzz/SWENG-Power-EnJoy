In this section we provide several different scenarios, each of which should clarify how the user interacts with the system in practical situations.

\subsubsection{Scenario 1 - User registration and login}
Robert is a tech fan and always wants to try out new things. He just discovered PE and immediately downloaded PEM. Next, he inserted both his credentials, payment informations and the numbers of his driving license and his ID card. Robert is also asked to read and accept the terms and conditions. He followed the procedure and after a while he received an email with a password that he must use to access PE system. Robert can now log in with his email and password and search for available cars.

\subsubsection{Web app}
George is out of internet on this mobile phone. Fortunately, he is at home and can log in to PE system through his computer. He found a nearby car available and reserved it. Then, he reached the selected car and sent a SMS with his phone to PE. After a while, the car unlocked and George used it.


\subsubsection{Scenario 2 - }
Anna is a university student and currently studies abroad. It is a long time she had not seen her parents so she decided to surprise them by getting home. She landed in the nearest airport and took a train to the city. Anna knows that her house is far from the metro, so she decided to use PE service and get a car. She installed PEM in her phone and logged in. She started searching for an available car within 5 minutes walk from the station and luckily she found a nearby parking area with an available car with 70\% charge - enough for her trip. She did the reservation and headed toward the car. When Anna was close to the car, she unlocked the car by pressing the 'unlock' button on PEM. While she was driving, the display in the car was showing her all nearby safe areas and charging information. Anna parked the car in the closest safe area and got off the car. 

\subsubsection{Scenario 3}
Bob is an environmentalist and takes very seriously the issue of climate change. To get around the city he usually uses his electric car but unfortunately he forgot to charge the battery the day before. He downloaded the application on his mobile phone and signed up to the system by providing personal and banking information. He searched for available car in 5 km distance and found one, so he immediately reserved the car and started walking. When he arrived close to the car 15 minutes later, he opened the mobile app and he sent a notification to the system to unlock the car. He chose the destination and reserved a parking in that area through the touch screen computer situated in the car. Once he arrived in his destination, he noticed on the car display that the battery level was under 50\%, so he plugged the car into the power grid. He suddently received a notification on his mobile phone with a 30\% discount and an acknowledgment for his good action.

\subsubsection{Scenario 4}
Carl is always in a hurry and hates losing any minute of his precious time. For this reason, he reserved a car through the mobile application while he was still at work, in order to be ready to pick it up 30 minutes later and get home early that day. Unfortunately his boss called him for an unexpected meeting that lasted 2 hours. Carl forgot to cancel his previous reservation, so the system charged him a fee and made the reserved car available again.

\subsubsection{Scenario 5}
Dave enjoys going out on Saturday nights and come back home as late as possible with his two friends, Mario and John.  Neither he and his friends has his own car, so they usually use public transportation to get to the pubs. Unfortunately, public transportation service in the city is not available late at night. Therefore, when they wanted to get home, Dave signed in with his account and reserved the closest car available. When they got in the car, the car detected Mario and John as passengers while Dave was driving all of them back home. Dave brought each of his friends at their home and parked the car in a safe area. Soon after he get out the car, he received his check and a notification of the applied discount.

\subsubsection{Scenario 6}
Elaine is very late for an exam and she urgently need a car to get to university. She found an available car, reserved it and drove to her destination. She had no time to park the car in a safe area, hence she left the car in the first parking she found and ran to take the exam. The car got locked as soon as the door was closed, but it continued charging the user because it was not parked in a safe area. After she finished her exam, Elaine returned to the car and unlocked it by sending a notification to the system through her mobile phone. She parked the car in the closest safe area and plugged in to the power grid. She obtained a discount for this action for the entire time she kept the car unavailable, but still the bill to pay was salty. Poor Elaine, hope she had passed the exam.

\subsubsection{Scenario 7}
Fred the Trouble Maker is always very unlucky. He signed in the system with his mobile phone and picked up a car to get home. He was happily driving down main street when a truck turned suddenly and struck the car. Fred, luckily undamaged, suddenly phoned to the call center of PE system and reported the accident. After a while, Simon the operator came by and helped Fred with all the required procedure. Since the car was destroyed, Simon gave a ride home to Fred.

\subsubsection{Scenario 8}




